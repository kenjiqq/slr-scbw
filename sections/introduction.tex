% Introduction here
%\textbf{this section needs a rewrite}

This document details the literature review performed by the Starcraft AI team at NTNU during our specialization project.  The Starcraft team consists of twelve students, all of whom share a single assignment:

\emph{Create an architecture for an agent which plays Starcraft.}

Because we all share this assignment, we decided to share the work of searching for relevant literature.  One of our supervisors described the Structured Literature Review, a method which allowed us to distribute the work, and find literature which is relevant to our assignment.





%--------------------------------
%The Structured Literature Review is very easy to distribute due to the relative disjointedness of the subproblems of each phase. There is also the fact that when we are this many people, actually completing the review in the limited time we have available becomes feasible.



%A Structured Literature Review (SLR) is a method for finding relevant literature for a specific research question. Instead of doing an exploratory search, that is, just searching randomly for relevant information, you define search terms, and where to search for information. That way, you can be more sure of that you found the relevant literature in the places you searched. 

%This does however mean that you do not find the relevant information that may be placed in other places that it is possible to search in. The reason this is acceptable is that there is a tradeoff between how much time you should spend searching for literature, and what payoff you get. With SLR you can set a goal for when you are done looking for literature, and when you reach that goal you can begin researching the question you wanted an answer for. Doing just an exploratory search you may end up never getting done, as it is hard to know when you are done searching for a particular term, and in a specific place.

%Another benefit to doing an SLR is that every step in it is documented, so if someone wants to review your work they can see where you looked for information, and understand why you chose the articled you chose. It also reduces the amount of work someone else trying to solve the same problem as you has to do, as they can look at your SLR, and use it to choose what literature to use, without having to do the search themselves. 
