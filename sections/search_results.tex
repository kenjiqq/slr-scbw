% Search results here

\textbf{Towards conscious-like behavior in computer game characters}\\
\textit{Authors:} Arrabales, Ra\'{u}l; Ledezma, Agapito; Sanchis, Araceli\\
\textit{Reviewed by:} Dag {\O}yvind\\
\textit{Quality score:} 6,67\\
\textit{Review:} The article describes the architecture Cera-Cranium, a cognitive architecture.  We see influences from many other forms of architectures, but this adds a twist in the form of ``focus of attention''.  This allows the architecture to reason along several parallel threads, and then select objectives to focus on.  In the article, the main focus is on achieving conscious-like behavior (as per the title) in FPS-bots.  The architecture was however developed for applications in robotics, with a strong focus on adaptability to other domains.  Due to this, we believe it may be applicable to our RTS-domain.
%Tags: cognitive architecture, conscious behaviour, FPS


\textbf{An Architecture for Autonomy}\\
\textit{Authors:}: Alami, R; Chatila, R; Fleury, S; Ghallab, S; Ingrand, F\\
\textit{Reviewed by:} Dag {\O}yvind\\
\textit{Quality score:} 6,90\\
\textit{Review:} The article provides a detailed description of a three-layer architecture for robot control.  First, we are given a global view of the architecture, the layers, along with their responsibilities and a rationale for their inclusion in the architecture.  The authors then continue with explanations of each layer, how to implement them, issues which may arise and finally their own implementation in a sample domain.  The last third of the article then gives further details in the form of a complete example using mail delivery robots.  While the article mainly focuses on robot control, the architecture should easily be transferrable to our domain, and has facilities for both reactive control and higher level constructs such as learning.
%Tags: layered architecture, robot control, reactive control, high level reasoning

\textbf{Non-communicative multi-robot coordination in dynamic environments}\\
\textit{Authors:}: Kok, J; Spaan, M; Vlassis, N\\
\textit{Reviewed by:} {\O}ystein\\
\textit{Quality score:} 7,50\\
\textit{Review:} The paper is about coordination of several autonomous agents (robots) playing RoboSoccer, by means of coordination graphs (CGs), ie. local constraints between subsets of the agents. The agents can only rely on their own noisy perception to sense the world state and cooperate, which is quite different from an RTS setting. There is also focus on coordination without communication, ie. predicting what actions the other agents will most likely take. There is some talk of game theory.  If implementing the RTS bot as a MAS, there might be something of interest here.
%Tags: Cooperative MAS, coordination, coordination graphs, action prediction, RoboSoccer

\textbf{Modeling interaction strategies using POS: An application to soccer robots}\\
\textit{Authors:}: Koning, Jean-Luc; Oudeyer, Pierre-Yves\\
\textit{Reviewed by:} {\O}ystein\\
\textit{Quality score:} 6,79\\
\textit{Review:} Not really about agent architecture, but about how to separate the communication protocol from agent architecture in a cooperative multi-agent system, ie. the agents need to have communication ability, which can be applied to any custom protocol. The example domain is RoboSoccer. There are basically a set of robots playing soccer, each of which is an autonomous agent, and the agents communicate with each other. If implementing the RTS bot as a MAS, there might be something of interest here.
%Tags: Cooperative MAS, communication protocols

\textbf{Transfer Learning of Hierarchical Task-Network Planning Methods in a Real-Time Strategy Game}\\
\textit{Authors:}: Lee-Urban, Stephen; Mu\~{n}oz-avila, H\'{e}ctor; Parker, Austin; Kuter, Ugur; Nau, Dana\\
\textit{Reviewed by:} {\O}ystein\\
\textit{Quality score:} 9,64\\
\textit{Review:} This paper concerns management of the hierarchy of tasks and subtasks in a game. The game domain used is an RTS called ``MadRTS''. The main focus is on integrating planning and learning of HTN methods, ie. when different methods are applicable. An architecture for this is presented. Interesting article with potentially high relevance for our domain.
%Tags: HTN, planning, learning

\textbf{An explainable artificial intelligence system for small-unit tactical behavior}\\
\textit{Authors:}: Lent, Michael van; Fisher, William; Mancuso, Michael\\
\textit{Reviewed by:} {\O}ystein\\
\textit{Quality score:} 6,43\\
\textit{Review:} Describes the AI system of a military training software named ``Full Spectrum Command'', which is something like a first person RTS. It focuses only on the tactical aspect, ie. no resource gathering or unit building, but it does describe a potentially useful architecture for modularizing the AI, although the description is rather superficial. The main focus is on the software's ability to explain the reasoning of the AI to the user.
%Tags: Explanation, modular AI, FP-RTS

\textbf{Program Steering : Improving Adaptability and Mode Selection via Dynamic Analysis}\\
\textit{Authors:}: Lin, Lee Chuan\\
\textit{Reviewed by:} {\O}ystein\\
\textit{Quality score:} 7,50\\
\textit{Review:} MSc dissertation concerning the use of a kind of machine learning technique to improve program steering. Program steering means that the agent can be in one of a number of different modes, and we are concerned with steering it into the most appropriate mode in a given situation. The conditions for beeing in a mode are preprogrammed by the developer. A machine learning technique is used for selecting mode in situations that were not thought of by the developer. Training cases are chosen from test runs where the agent acts correctly. Later the agent uses these data to select the most appropriate mode in any situation, even though the particular situation may not have been thought of in the original program. It is applied to a couple of domains, including a sort of RTS which consists of several completely autonomous agents that need to sense and act (noisily) on their own and try to cooperate, ie. there is no omniscient ``overmind''.
%Tags: Program steering, mode selection, RTS, MAS

\textbf{Enhanced NPC behaviour using goal oriented action planning}\\
\textit{Authors:}: Long, Edmund\\
\textit{Reviewed by:} {\O}ystein\\
\textit{Quality score:} 7,50\\
\textit{Review:} MSc dissertation which compares the performance (both in terms of gameplay proficiency and system resource usage) of GOAP and FSM in shooter games, concluding that GOAP is far superior, although with some disadvantages. Also states in the conclusion that GOAP should be even more ideal for RTS games, although this claim is not supported by empirical evidence. Source code for the implemented system is available at \url{http://www.edmundlong.com}.
%Tags: GOAP (vs FSM), shooter games (FPS)


\textbf{A few good agents: Multi-agent social learning}\\
\textit{Authors:}: Oh, Jean, Smith, S.F.\\
\textit{Reviewed by:} Jan\\
\textit{Quality score:} 5.12\\
\textit{Review:} Is a hybrid solution between nash equilibrium and central admin for agent cooperation. Does have a figure about the agent architecture for action selection, but very little description. Could be useful for someone who wants to design agents to do the jobs.

\textbf{State of the art report and requirement specification}\\
\textit{Authors:}: St\'{e}phane Donikian, Nicolas P\'{e}pin\\
\textit{Reviewed by:} Jan\\
\textit{Quality score:} 8.33\\
\textit{Review:} Fairly long (50 pages), but covers many interesting architecture topics in optics that can be very useful for our domain. Points of interest:
Chapter 2-4: General walkthrough of action selection, and layered approaches
Chapter 5: Navigation (for troop movement in our application)

\textbf{Automatic fuzzy decision making system with learning for competing and connected businesses}\\
\textit{Authors:}: Oderanti, Festus Oluseyi, De Wilde, Philippe\\
\textit{Reviewed by:} Jan\\
\textit{Quality score:} 7.50\\
\textit{Review:} Presents a fuzzy logic system that they have tested extensively in competing and connected businesses. Explains a bit on architecture they used, but is mostly a case study for a Board game and a Network game.

\textbf{Symbolic representation of game world state: Toward real-time planning in games}\\
\textit{Authors:}: Orkin, Jeff\\
\textit{Reviewed by:} Jan\\
\textit{Quality score:} 4.64\\
\textit{Review:} Bit short, but offers a high level architecture description of GOAP in games. Could be more detailed, and no figures. 

\textbf{Undergraduate Dissertation: A critical analysis of Behaviour- Oriented Design (BOD), based on experiences in using it to create an Unreal Tournament Capture-the-Flag (CTF) team.}\\
\textit{Authors:}: Partington, Samuel J.\\
\textit{Reviewed by:} Jan\\
\textit{Quality score:} 7.50\\
\textit{Review:} Long dissertation (118 pages). Is an analysis of the BOD architecture in Unreal Tournament. Chapter 2 describes a Behaviour oriented design, chapter 3 describes other similar possible architectures that could be worth looking at.

\textbf{Threat Analysis Using Goal-Oriented Action Planning Planning in the Light of Information Fusion}\\
\textit{Authors:}: Bjarnolf, Philip\\
\textit{Reviewed by:} Jan\\
\textit{Quality score:} 8.33\\
\textit{Review:} Master thesis about GOAP. Gives a very good explanation and diagrams of the GOAP architecture (chapter 2), and some explanation of finite state machines and blackboard structures. 

\textbf{Behavioural State Machines: Agent Programming and Engineering}\\
\textit{Authors:}: Nov\'{a}k, Peter\\
\textit{Reviewed by:} Jan\\
\textit{Quality score:} 8.33\\
\textit{Review:} This is a doctoral thesis (200 pages), so i didnt have time to do more than scratch the surface. Chapter 3 describes a modular BDI (Belief-Desire-Intention) architecture, that could be worth having a look at.


\textbf{Practical development of Goal-Oriented Action Planning AI}\\
\textit{Authors:}: Pittman, D.L.\\
\textit{Reviewed by:} Jan\\
\textit{Quality score:} 6.79\\
\textit{Review:} A very thorough thesis about making a Goal Oriented Action Planner in a FPS game. Though it doesnt contain fancy boxes and arrows, an architecture can be deduced from the text. It is more focused on the debugging and algorithms needed though.

\textbf{A Teamwork Infrastructure for Computer Games with Real-Time Requirements}\\
\textit{Authors:}: Monteiro, Ivan,Alvares, Luis, Dignum, Frank, Bradshaw, Jeff, Silverman, Barry, van Doesburg, Willem\\
\textit{Reviewed by:} Tobias\\
\textit{Quality score:} 7.98\\
Describes the tool TWproxy built to facilitate teamwork between multiple agents in complex environments. Used to guide bot behaviour in the first person shooter unreal tournament. Shows that twproxy helps the bots perform better with shared strategies and teamwork. Comparisons with bots using other cooperation tactics, no cooperation and against human players.

\textbf{Agents for Massive On-line Strategy Turn Based Games}\\
\textit{Authors:}: Ribeiro, Luis Miguel Landeiro\\
\textit{Reviewed by:} Tobias\\
\textit{Quality score:} 8.81\\
Master thesis, approximately 80 pages, introduces artificial intelligence to a massive mutiplayer turn based strategy game. chapter 2 describes agent architectures, chapter 3 discusses different case studies on how artificial intelligence has been implemented in different games. chapter 5 introduces the proposed agent architecture, a hybrid horisontal approach based on subsumption.

\textbf{An Extended Behavior Network for a Game Agent : An Investigation of Action Selection Quality and Agent Performance in Unreal Tournament}\\
\textit{Authors:}: Corr\^{e}a, Silva, Alvares, Luis Ot\'{a}vio\\
\textit{Reviewed by:} Tobias\\
\textit{Quality score:} 5.6\\
Uses Extended behaviour networks in agents playing Unreal tournament. Shows superior performance to purely reactive bots and bots built on finite state machines.



\textbf{Behavior Modeling and Real-Time Simulation for Autonomous Agents using Hierarchies and Level-of-Detail}\\
\textit{Authors:}: Niederberger, Christoph Beat\\
\textit{Reviewed by:} Tobias\\
\textit{Quality score:} 8.33\\
Doctoral thesis,approximately 200 pages, presents an agent architecture for realtime agents in dynamic environments. chapter 2, 3 and 4 presents and discusses different agent architectures, chapter 5 discusses handling the complexity in the system proposed. chapter 6 discusses implementational details, proposes a blackboard architecture.

\textbf{Efficient, realistic NPC control systems using behavior-based techniques}\\
\textit{Authors:}: Khoo, Aaron, Dunham, Greg, Trienens, Nick,Sood, Sanjay\\
\textit{Reviewed by:} Tobias\\
\textit{Quality score:} 5.85\\
Argues that advanced ai techniques are overkill for artificial intelligence in games, and that behavioural methods in this case finite state machines are sufficient to create agents that act human. no experiments, can show a that bots use a very small amount of memory, uses  anecdotal evidence.

\textbf{Reasoning And Learning For Intelligent Agents}\\
\textit{Authors:} Sioutis, Christos\\
%Tags: cognitive, reasoning, bdi, ooda\\
\textit{Reviewed by:} J{\o}rgen\\
\textit{Quality score:} 7.62\\
\textit{Review:} This PhD on 188 pages covers the problem of implementing intelligent agents into complex real-time enviornments, focusing on cognitive (hybrid) reasoning, learning, and overall design of an agent-oriented application.
The author presents a design for implementing for a learning-agent system of bots for Unreal Tournament capture the flag. The design is an BDI architecture which retains elements from the OODA loop (a military combat strategy concept, or process model - observe, orient, decide, and act.). 


\textbf{A Method for Generating Emergent Behaviors using Machine Learning to Strategy Games}\\
\textit{Authors:}: Machado, Alex F. V.Clua, Esteban W.Zadrozny, Bianca\\
%Tags: rts, uml,machine learning
\textit{Reviewed by:} J{\o}rgen\\
\textit{Quality score:} 6.07\\
\textit{Review:} Proposes an architecture paradigm for development of games with machine learning. The paper illustrates the architecture with a UML diagram, implements an RTS with it, and does measures of the performance of various machine learning algorithms. Unfortunately the focus is mainly on a game architecture as a whole and learning algorithms. The paper is highly practical, but lacks depth on the architectural part that we are interested in.

\textbf{Two Case Studies for Jazzyk and BSM}\\
\textit{Authors:} Michael, Nov, Mainzer\\
\textit{Reviewed by:} Espen\\
\textit{Quality score:} 6.31\\
\textit{Review:} This paper defines an architecture used by BDI agents. The framework is called Behavioural state machine. It is implemented in a FPS and a ``Puck''-robot.
%Tags: Belief Desire Intention, Behavioural state machine, FPS, 

\textbf{Personality-based adaption for teamwork in game agents}\\
\textit{Authors:} Tan, Cheng\\
\textit{Reviewed by:} Espen\\
\textit{Quality score:} 8.81\\
\textit{Review:} Outlines a cognitive architecture. The paper focuses on adaptive playing styles and is implemented in an FPS environment created for the experiment. The architecture is not defined in detail, but the components explained. 
%Tags: Cognitive Architecture, FPS, Adaptive

\textbf{Hierarchical controller learning in a first-person shooter}\\
\textit{Authors:} Van Horn, N. et. al.\\
\textit{Reviewed by:} Espen\\
\textit{Quality score:} 8.81\\
\textit{Review:} Defines a hierarchical architecture for controlling a FPS-bot. Trains its modules with a combination of NN and EA, and each part is both trained separately and together. The testing is done in Unreal Tournament 2004. 
%Tags: Hierarchical Architecture, FPS, Unreal Tournament 2004

\textbf{Computer intelligence in strategy games}\\
\textit{Authors:} Vemuri, V.R. et. al.\\
\textit{Reviewed by:} Espen\\
\textit{Quality score:} 5.6\\
\textit{Review:} Talks about a cognitive architecture that plays a strategy game and models emotion, however it is not explained in detail and no experiments are done.
%Tags: Cognitive Architecture, Strategy Game

\textbf{Petri Net Plans}\\
\textit{Authors:} Ziparo, V.A. et. al.\\
\textit{Reviewed by:} Espen\\
\textit{Quality score:} 6.43\\
\textit{Review:} Explains a way of making plans for multiple mobile robots. Might be useful if it is incorporated in an architecture.
%Tags: Multi-robot, Planning

\textbf{Sensing-based shape formation on modular multi-robot Systems: A theoretical study}\\
\textit{Authors:} Yu, Chih-han, Nagpal, R.	\\
\textit{Reviewed by:} Espen\\
\textit{Quality score:} 10\\
\textit{Review:} Defines a modular architecture for controlling a robot, where each module is a bodypart. Works in real time. Uses a control algorithm as means to decide who does what. A networked multiagent system.
%Tags: Modular, Architectrure, Networked multiagent system

\textbf{Intelligent online case-based planning agent model for real-time strategy games}\\
\textit{Authors:} Fathy, Ibrahim, Aref Mostafa, Enayet Omar, Al-Ogail Abdelrahman	\\
\textit{Reviewed by:} Ken\\
\textit{Quality score:} 6.55\\
\textit{Review:} The goal of this paper is to combine an online cased-based planing method with reinforcement leaning to improve the effectiveness and make it able to learn on the fly during games. The model is evaluated using Wargus. But it is not implemented in this paper. The paper presents the technique for integrating the learning, the architecture of it. But it is not a  very module architecture without some changes, and the paper mentions that this could be used as some sort of commander, with modules for the rest.

\textbf{Agent oriented control in Real-time Computer Games}\\
\textit{Authors:}: Behrens, T\\
\textit{Reviewed by:} Stian\\
\textit{Quality score:} 10\\
\textit{Review:} A good paper detailing the use of MAS in the domain of real-time strategy games. It describes how to use agents (and agent oriented programming) in the different problems that needs to be solved to successfully play real-time strategy games. It also describes the various entities and artifacts that make up an RTS game and how the agent(s) interact with them. 

\textbf{Using Multi-Agent Potential Fields in Real-Time Strategy Games}\\
\textit{Authors:}: Hagelb\"{a}ck, Johan and Johansson, Stefan J.\\
\textit{Reviewed by:} Stian\\
\textit{Quality score:} 10\\
\textit{Review:} A good paper describing how a MAS can utilize potential fields to make decisions. It describes the approach of designing an agent using potential fields step by step. It also discusses the potential of using potential fields in the domain of real-time strategy games.

\textbf{MARS, A Multi-Agent System Playing RISK}\\
\textit{Authors:}: Johansson, Stefan J.\\
\textit{Reviewed by:} Stian\\
\textit{Quality score:} 7.5\\
\textit{Review:} An implementation using MAS in the domain of the turn-based strategy game RISK. Details the architecture and design of the agent. The agent is then pitted against other bots playing the same game and the results are presented. Overall a nice example of MAS in action. 

\textbf{A Multi-Agent Architecture for Game Playing}\\
\textit{Authors:}: Kobti, Ziad and Sharma, Shiven\\
\textit{Reviewed by:} Stian\\
\textit{Quality score:} 8.33\\
\textit{Review:} A paper detailing the design and architecture of an agent capable of general game playing. Features a system that uses evolutionary algorithms within a MAS which is used to play simple adverserial turnbased games (like tic tac toe). The design and approach to solve the problem is not applicable to the domain of real-time strategy games because capitalizes heavily on the reduction of the problem of game playing to simple turnbased games.

\textbf{Real-Time Plan Adaption for Case-Based Planning in Real-Time Strategy Games}\\
\textit{Authors:}: Sugandh, Neha\\
\textit{Reviewed by:} Stian\\
\textit{Quality score:} 8.33\\
\textit{Review:} A paper detailing an agent that used Case-based Planning to play the real-time strategy game of Wargus. It features a detailed description of the architecture of the system. The paper has a large section devoted to how plans are made, executed and adapted during execution. Adaption of plans is the main focus of this paper, allthough not directly tied to architecture this is a very important aspect when using the approach of case-based techniques for playing real-time strategy games.

\textbf{A Combined Tactical and Strategic Hierarchial Learning Framework in Multi-agent Games}\\
\textit{Authors:}: Tan, Chek Tien and Cheng, Ho-lun\\
\textit{Reviewed by:} Stian\\
\textit{Quality score:} 9.5\\
\textit{Review:} A paper outlining a framework for organizing agents in teams which cooperates to beat an opposing team. The framework includes a commander-agent with the capabilities of planing actions and communicating them to its team. Implemented and tested in a strategic first person shooter, which has many of the same problem areas as a real-time strategy game.
 
\textbf{Unreal GOAL Bots - Conceptual Design of a Reusable Interface}\\
\textit{Authors:}: Hindriks, van Riemsdijk, Behrens, Korstanje, Kraayenbrink, Pasman, de Rijk\\
\textit{Reviewed by:} Robin\\
\textit{Quality score:} 7.62\\
\textit{Review:} This paper argues that an interface between real-time systems, exemplified by the game Unreal Tournament 2004 and Belief-Desire-Intention agents would be a useful tool and goes on to describe such an interface. The article outlines a system roughly divided into 3 parts, the GOAL Interpreter for a BDI Agent scripting language, EIS which contains subsystems specific to the game and the game itself. It rounds off by discussing the results of training students to write agents with the system.

\textbf{Discovering Abstract Concepts to Aid Cross-Map Transfer for a Learning Agent}\\
\textit{Authors:}: Herpson, Corruble\\
\textit{Reviewed by:} Robin\\
\textit{Quality score:} 7.26\\
\textit{Review:} Transfer of knowledge is the central theme in this article, with an illustrative example being that knowledge learned on one map in a strategy game should not have to be discarded upon playing a different map. It focuses on developing the knowledge base part of the architecture of a game-playing agent using reinforcement learning and / or case based reasoning.

\textbf{MADeM: a multi-modal decision making for social MAS}\\
\textit{Authors:}: Grimaldo, Lozano, Barber\\
\textit{Reviewed by:} Robin\\
\textit{Quality score:} 7.26\\
\textit{Review:} This paper presents an interesting discussion on how to make multiple agents make decisions that are ``socially acceptable'', that is decisions that satisfy all agents in the system. It discusses the architecture of a system capable of simulating multiple agents having social interactions in a virtual 3D environment. 

\textbf{Nested Look-Ahead Evolutionary Algorithm Based Planning for a Believable Diplomacy Bot}\\
\textit{Authors:}: Kemmerling, Markus et. al.\\
\textit{Reviewed by:} Alek	\\
\textit{Quality score:} 7.98\\
\textit{Review:} Improves an existing Diplomacy (a board game) bot with a genetic algorithm. Diplomacy is played real-time, in that all players execute their moves simultaneously. Shows that EA is a good alternative because of the complexity of the game. Implements planning, one turn into the future. Comparison with State of the Art bot. Proves it has been significantly improved.

\textbf{Real-Time Neuroevolution to Imitate a Game Player}\\
\textit{Authors:}: Ki, Hyun-woo et. al.\\
\textit{Reviewed by:} Alek\\
\textit{Quality score:} 9.17\\
\textit{Review:} Using a neuroevolution (NE) algorithm to imitate a human player's playstyle on a simplified Starcraft clone. Evolution of neural networks with online learning. Only moving and battling of units are implemented. Shows experimentally that bot shows human-like behavior. Highly interesting for a bio-inspired solution.

\textbf{Evolving Bot AI in Unreal} \\
\textit{Authors:}: Mora, Antonio Miguel et. al.\\
\textit{Reviewed by:} Alek\\
\textit{Quality score:} 8.33\\
\textit{Review:} Improving the default bot in the game Unreal with a genetic algorithm and genetic programming. Argues the original bot in Unreal is still state of the art. Shows experimentally they managed to improve it using EA and GA.

\textbf{Intelligent AI Agents Using Planning} \\
\textit{Authors:}: Stephen Flockton\\
\textit{Reviewed by:} Alek\\
\textit{Quality score:} 7.38\\
\textit{Review:} Argues the use of planning in computer games. Explains planning and the history of planning in computer games, and its advantages and disadvantages. Argues state of the art planning bots are superior. Describes the implementation of a planner.

\textbf{An explainable artificial intelligence system for small-unit tactical behavior}\\
\textit{Authors:}: Lent, Michael Van; Fisher, William; Mancuso, Michael\\
\textit{Reviewed by:} Magnus\\
\textit{Quality score:} 5.36\\
\textit{Review:} This article describes the AI architecture and explanation capability of a training system developed for the U.S. Army. The traning system is an RTS-like game, it is strategy in real time, but not very similar to normal commercial RTS games. It has an explanation system that explains why the AI did certain actions, and explains what happened, to be used in military training. This article focuses mostly on explanation, that can be used to understand why a bot does the thing it does.

\textbf{Transfer learning in real-time strategy games using hybrid CBR/RL}\\
\textit{Authors:}: Manu Sharma; Michael Holmes; Juan Santamaria; Arya Irani; Charles Isbell; Ashwin Ram\\
\textit{Reviewed by:} Magnus\\
\textit{Quality score:} 8.33\\
\textit{Review:} This article presents an RTS bots that uses a mix of CBR and Reinforcment learning. The RTS in question is MadRTS. The article gives a good overview of the architecture, and is a good read if you are interested in using CBR for an RTS bot. It is using a layered approach, but does not seem very suited for collaboration between several persons/groups.

\textbf{Reactive planning idioms for multi-scale game AI}\\
\textit{Authors:}: Ben G. Weber; Peter Mawhorter; Michael Mateas; Arnav Jhala\\
\textit{Reviewed by:} Magnus\\
\textit{Quality score:} 9.29\\
\textit{Review:} This article presents the architecture and design decisions of a Starcraft bot. The bot was participant at AIDEE 2010, under the name of EISBot. Highly modular architecture with good description of it. Using managers for parts of gameplay that can be developed individually it seems.
