% Structured Literature Review here

A Structured Literature Review (SLR) is a method for a thorough and structured search for literature relevant to a specified research question \cite{Kofod2011}.  The goal of this method is to ensure the quality and the coverage of the literature related to the research question, and to be reproducible.

Before one can start the search and review activities, one must have a clearly defined research question.  This is codified in the SLR as a problem P with additional constraints C.  These provide guidelines when one constructs the search terms.

The SLR consists of three phases. The phases covers the initial collection of literature, initial filtering and full text scrutiny.  The procedure and results of each phase should be documented and reproducible.  In the first phase, one constructs a search term of the form group1 AND group2 AND .. AND groupN.  Each group is structured as term1 OR .. OR termN.  This query is submitted to several search engines, and the documents returned are passed to the next phase.  During the second phase, all documents are filtered according to a rough criteria, usually based on information available in the abstract and conclusion of a document.  In this phase, documents which are clearly false positives should be removed.  The remaining documents are candidates for full-text screening.  The last phase consists of reading the full document, and filtering them according to predetermined criteria.  The remaining documents should now serve as a basis to answer the original research question.

The biggest advantage of this method is that the result of the review is well documented and reproducible, covering a very wide range of sources and material. However, it might be that you do not find all the relevant information. This is because you chose sources according to research problem, and information may be found in other sources than the ones searched in.